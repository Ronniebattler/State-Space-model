\documentclass[UTF8,12pt]{ctexart}
%加载包
\usepackage{ctex}
\usepackage{geometry}%排版
%a4版面,页边距1英寸,showframe 增加边框的参数。
% 设置为A4纸,边距适中模式(永中office)
\geometry{%
	width = 210mm,%
	height = 297mm,
	left = 31.8mm,%
	right = 31.8mm,%
	top = 25.4mm,%
	bottom = 25.4mm%
}

%\hyphenpenalty = 1000% 断字设置,值越大,断字越少。
%\setlength{\parindent}{2em}% 缩进
%\setlength{\parskip}{0.5ex} % 段间距

\usepackage{cite} %引用
\usepackage{amsmath}
\usepackage{amsfonts}
\usepackage{amssymb}%公式
\usepackage{amsthm}%定理环境

%\usepackage{ntheorem}%定理环境,使用这个会使\maketitle版式出问题
\usepackage{bm}%加粗

\usepackage{mathrsfs}
\numberwithin{equation}{section}%对公式以节{section}为基础进行编号.变成(1.1.1)有chapter才有1.1.1,不然只有section是1.1
%\theoremstyle{plain}%定理用latex默认的版式
\newtheorem{thm}{Theorem}[section]
%\theoremstyle{definition}%定义用definition格式
\newtheorem{defn}{Definition}
%\theoremstyle{remark}%用remark格式
\newtheorem{lemma}[thm]{lemma}
\newtheorem{example}{Example}[section]

\usepackage{multirow}%表格列合并宏包,\multirow命令.

\usepackage{tabularx}%表格等宽,\begin{tabularx}命令.

 
%盒子
\usepackage[many]{tcolorbox}    	% for COLORED BOXES (tikz and xcolor included)
\usepackage{setspace}               % for LINE SPACING
\usepackage{multicol}               % for MULTICOLUMNS
%自定义设定		
	\definecolor{main}{HTML}{5989cf}    % setting main color to be used
	\definecolor{sub}{HTML}{cde4ff}     % setting sub color to be used
	
	\newtcolorbox{boxF}{
		colback=blue!5!white,
		enhanced,
		boxrule = 1.5pt, 
		colframe = white, % making the base for dash line
		borderline = {1.5pt}{0pt}{main, dashed} % add "dashed" for dashed line
	}
\tcbuselibrary{skins, breakable}% 支持文本框跨页

\usepackage[english]{babel}% 载入美式英语断字模板

\usepackage{graphicx}
\usepackage{float}
\usepackage{subfigure} %插入多图时用子图显示的宏包

\usepackage{algorithm,algorithmic}%算法

\usepackage{listings}   %代码块
\usepackage{xcolor}
\definecolor{codegreen}{rgb}{0,0.6,0}
\definecolor{codegray}{rgb}{0.5,0.5,0.5}
\definecolor{codepurple}{rgb}{0.58,0,0.82}
\definecolor{backcolour}{rgb}{0.95,0.95,0.92}
%设置代码块
%\lstdefinestyle{mystyle}{
%	backgroundcolor=\color{backcolour},   
%	commentstyle=\color{codegreen},
%	keywordstyle=\color{magenta},
%	numberstyle=\tiny\color{codegray},
%	stringstyle=\color{codepurple},
%	basicstyle=\ttfamily\footnotesize,
%	breakatwhitespace=false,         
%	breaklines=true,                 
%	captionpos=b,                    
%	keepspaces=true,                 
%	numbers=left,                    
%	numbersep=5pt,                  
%	showspaces=false,                
%	showstringspaces=false,
%	showtabs=false,                  
%	tabsize=2
%}

%\lstset{
%	language=R,                     % 设置语言
%	numbers=left,                   % 在左侧显示行号
%	numberstyle=\tiny\color{gray},  % 设置行号的样式
%	commentstyle=\color{olive},     % 注释样式
%	keywordstyle=\color{blue},      % 关键字样式
%	stringstyle=\color{red},        % 字符串样式
%	basicstyle=\ttfamily,           % 基本代码样式
%	breakatwhitespace=false,        % 自动断行
%	breaklines=true,                % 自动断行
%	captionpos=b,                   % 设置标题位置
%	keepspaces=true,                % 保持格式
%	showspaces=false,               % 显示空格
%	showstringspaces=false,
%	showtabs=false,
%	tabsize=2
%}

\lstset{
	language=R,
	numbers=left,
	numberstyle=\tiny\color{gray},
	commentstyle=\color{olive},
	keywordstyle=\color{blue},
	stringstyle=\color{red},
	basicstyle=\ttfamily,
	breakatwhitespace=false,
	breaklines=true,
	captionpos=b,
	keepspaces=true,  % 保持源代码中的空格
	showspaces=false,
	showstringspaces=false,
	showtabs=false,
	tabsize=1,
	xleftmargin=0em,
	xrightmargin=0em,
	numbersep=5pt  % 调整行号和代码之间的距离
}

\usepackage{appendix}%附录

\usepackage{hyperref}%可以生成pdf书签,可以跳转
\hypersetup{
	colorlinks=true,
	linkcolor=black,
	citecolor=black,
}%使得目录没有红框 参考文献引用没有颜色

%侧栏笔记
\usepackage{marginnote}
\setlength{\marginparwidth}{2.8cm}%设置宽度
\renewcommand*{\marginfont}{\color{violet}\footnotesize}%fonts
%运用此命令就可加入侧栏笔记\normalmarginpar\marginnote{}

%图注
\usepackage{caption}

%参考文献
\usepackage[round]{natbib}


%画图
\usepackage{tikz}

%标题页
\title{Toy Notes of MCMC}
\author{Renhe W.}
\date{ }

%工具
%使用文本框
%\begin{tcolorbox}[enhanced]	\end{tcolorbox}
%代码框
%{\setmainfont{Courier New Bold}                       %设置代码字体                   
%\begin{lstlisting}

%\end{lstlisting}}

%文章开始部分

\begin{document}
	\captionsetup[figure]{labelfont={bf},labelformat={default},labelsep=period,name={图}}%设置图注
	
	\maketitle
	\tableofcontents%目录
	\listoffigures%图片目录
	\listoftables%表格目录
	\newpage
	\kaishu
	
	
	%------------------------------------------- 
	\section{MH算法}
	
	\subsection{MH算法基本步骤}
	\begin{tcolorbox}[enhanced]	
	
	1. 初始化: 选择一个初始状态 $x_0$,
	
	
	2. 对于每一步 $t=1,2, \ldots, T$
	
	- 建议步骤: 从建议分布 $q\left(x^{\prime} \mid x_t\right)$ 中抽取一个候选状态 $x^{\prime}$
	
	- 接受步骤: 以以下的接受概率 $A\left(x_t, x^{\prime}\right)$ 接受候选状态:
	$$
	A\left(x_t, x^{\prime}\right)=\min \left(1, \frac{p\left(x^{\prime}\right) q\left(x_t \mid x^{\prime}\right)}{p\left(x_t\right) q\left(x^{\prime} \mid x_t\right)}\right)
	$$
	如果接受,则 $x_{t+1}=x^{\prime}$ ,否则 $x_{t+1}=x_t$.
	
	这里, $p(x)$ 是我们想要采样的目标分布, $q\left(x^{\prime} \mid x_t\right)$ 是给定当前状态 $x_t$ 时,提议一个新状态 $x^{\prime}$ 的建议分布.
		
	\end{tcolorbox}

	
	\subsection{为什么这样能够工作?}
	
	关键在于保证马尔可夫链的平稳分布 (stationary distribution) 是我们要抽样的目标分布。 平稳分布 $\pi(x)$ 是一个分布,对其而言,如果我们从该分布中抽取一个样本并应用转移核,新 的样本仍然服从 $\pi(x)$ 。数学上我们可以表达为:
	$$
	\pi\left(x^{\prime}\right)=\sum_x \pi(x) P\left(x^{\prime} \mid x\right)
	$$
	这里, $P\left(x^{\prime} \mid x\right)$ 是从状态 $x$ 到状态 $x^{\prime}$ 的转移概率。
	$\mathrm{MH}$ 算法通过精心设计的接受准则确保了其转移核满足{\color{brown} 细致平稳条件},也就是说,对于任意的 状态 $x$ 和 $x^{\prime}$ ,以下等式成立:
	$$
	\pi(x) P\left(x^{\prime} \mid x\right)=\pi\left(x^{\prime}\right) P\left(x \mid x^{\prime}\right)
	$$	
	其中 $P\left(x^{\prime} \mid x\right)$ 是总的从状态 $x$ 到状态 $x^{\prime}$ 的转移概率,包括了提议和接受两个步骤,可以写 作:
	$$
	P\left(x^{\prime} \mid x\right)=q\left(x^{\prime} \mid x\right) A\left(x, x^{\prime}\right)
	$$
	通过MH算法的接受准则,我们可以验证细致平稳条件确实成立:
	$$
	\pi(x) q\left(x^{\prime} \mid x\right) A\left(x, x^{\prime}\right)=\pi\left(x^{\prime}\right) q\left(x \mid x^{\prime}\right) A\left(x^{\prime}, x\right)
	$$
	由于 $A\left(x, x^{\prime}\right)=\min \left(1, \frac{\pi\left(x^{\prime}\right) q\left(x \mid x^{\prime}\right)}{\pi(x) q\left(x^{\prime} \mid x\right)}\right)$ 和 $A\left(x^{\prime}, x\right)=\min \left(1, \frac{\pi(x) q\left(x^{\prime} \mid x\right)}{\pi\left(x^{\prime}\right) q\left(x \mid x^{\prime}\right)}\right)$ ,我们可以看 到这两边确实是相等的,从而确保了平稳分布 $\pi(x)$ 就是我们要采样的分布 $p(x)$.
	
	\subsection{Note}
	\subsubsection{为什么设置的是$	A\left(x_t, x^{\prime}\right)=\min \left(1, \frac{p\left(x^{\prime}\right) q\left(x_t \mid x^{\prime}\right)}{p\left(x_t\right) q\left(x^{\prime} \mid x_t\right)}\right)$}
	
	Metropolis-Hastings (MH) 算法中的接受概率
	$$
	A\left(x_t, x^{\prime}\right)=\min \left(1, \frac{p\left(x^{\prime}\right) q\left(x_t \mid x^{\prime}\right)}{p\left(x_t\right) q\left(x^{\prime} \mid x_t\right)}\right)
	$$
	是一个精心设计的准则,旨在确保生成的样本 $x$ 的分布最终收敛到目标分布 $p(x)$ 。这里的 $x_t$ 是当前状态, $x^{\prime}$ 是建议的下一个状态, $q\left(x_t \mid x^{\prime}\right)$ 是从状态 $x^{\prime}$ 到状态 $x_t$ 的转移概率, $p(x)$ 是我们想要采样的分布.
	
	理解这个接受概率的一个简单方法是考虑比率
	$$
	\frac{p\left(x^{\prime}\right) q\left(x_t \mid x^{\prime}\right)}{p\left(x_t\right) q\left(x^{\prime} \mid x_t\right)}
	$$
	
	分子 $p\left(x^{\prime}\right) q\left(x_t \mid x^{\prime}\right)$ : 这部分表示我们建议从状态 $x^{\prime}$ 移动到 $x_t$ 并且 $x^{\prime}$ 本身的概率。实际 上,这部分衡量了建议的状态 $x^{\prime}$ 到当前状态 $x_t$ 的“前进”概率.
	
	分母 $p\left(x_t\right) q\left(x^{\prime} \mid x_t\right)$ : 这部分表示我们建议从状态 $x_t$ 移动到状态 $x^{\prime}$ 并且 $x_t$ 本身的概率。这 部分衡量了当前状态 $x_t$ 到建议状态 $x^{\prime}$ 的“前进”概率.
	
	这个比率的直观意义是一个“平衡”:我们想要平衡从当前状态到建议状态的前进概率和反方向 的前进概率。
	\begin{itemize}
		\item 当这个比率大于1时,我们总是接受建议的状态,因为这意味着建议的状态比当 前状态更可能来自目标分布.
		\item 当这个比率小于1时,我们只有一定的概率接受建议的状态,这个概率正比于比率的大小.
	\end{itemize}
	\subsubsection{为什么引入均匀分布?}
	均匀分布在 $\mathrm{MH}$算法中的角色体现在决定是否接受建议状态的步骤. 具体来说,即使建议状态 的接受概率 $A\left(x_t, x^{\prime}\right)$ 小于1,我们仍然有可能接受它,这样做可以防止算法过早地陷入局部 最优解并增加探索性。我们通常通过以下方式使用均匀分布:
	\begin{enumerate}
		\item 从均匀分布 $U(0,1)$ 中抽取一个随机数 $u$.
		\item  如果 $u \leq A\left(x_t, x^{\prime}\right)$ , 接受建议状态,否则保持当前状态不变.
	\end{enumerate}
	
	使用均匀分布的这一步增加了算法的随机性,并允许它有可能接受一个在目标分布下不太可 能的状态,从而增加了算法的探索能力.

	\section{Gibbs sampling}
	Gibbs抽样是一种特殊的Metropolis-Hastings(MH)算法,其中提议分布是条件分布,而接受概率始终为1. 这意味着Gibbs抽样总是接受新提议的样本.
	
	\begin{tcolorbox}[title=Gibbs sampling]
		Gibbs抽样是一种在高维分布上进行抽样的方法。对于一个 $d$ 维分布 $p\left(x_1, x_2, \ldots, x_d\right)$ , Gibbs抽样在每一步依次对每一个变量 $x_i$ 进行抽样,条件于其他所有变量的当前值:
		$$
		x_i^{(t+1)} \sim p\left(x_i \mid x_1^{(t+1)}, x_2^{(t+1)}, \ldots, x_{i-1}^{(t+1)}, x_{i+1}^{(t)}, \ldots, x_d^{(t)}\right),
		$$
		这里, $(t)$ 是迭代的步数.
	\end{tcolorbox}
	

	Gibbs抽样作为MH算法的特例:
	
	为了理解Gibbs抽样是如何成为 $\mathrm{MH}$ 算法的一个特例的,我们需要考虑 $\mathrm{MH}$ 算法的两个主要步骤: 建议 (proposal) 和接受 (acceptance).
	\begin{itemize}
		\item 在MH算法中:
		\begin{itemize}
			\item 建议步骤: 从建议分布 $q\left(x^{\prime} \mid x\right)$ 中抽样一个候选样本.
			\item 接受步骤: 以一定的接受概率 $A\left(x, x^{\prime}\right)$ 接受这个样本.
		\end{itemize}
		\item 在Gibbs抽样中:
		\begin{itemize}
			\item 建议步骤: 从完全条件分布中抽样.
			\item 接受步骤: 总是接受从条件分布中抽样出的样本.
		\end{itemize}
	\end{itemize}
	如果我们将Gibbs抽样的建议步骤视为 $\mathrm{MH}$ 算法中的一个特殊情况,其中建议分布是条件分 布,那么Gibbs抽样的接受概率可以计算为:
	$$
	A\left(x, x^{\prime}\right)=\min \left(1, \frac{p\left(x^{\prime}\right) q\left(x \mid x^{\prime}\right)}{p(x) q\left(x^{\prime} \mid x\right)}\right)
	$$
	由于在Gibbs抽样中 $q\left(x^{\prime} \mid x\right)=p\left(x_i^{\prime} \mid x_{-i}\right)$ ,其中 $x_{-i}$ 表示除了 $x_i$ 之外的所有变量,我们可 以得出:
	$$
	A\left(x, x^{\prime}\right)=\min \left(1, \frac{p\left(x_i^{\prime} \mid x_{-i}^{\prime}\right) p\left(x_{-i}^{\prime}\right)}{p\left(x_i \mid x_{-i}\right) p\left(x_{-i}\right)}\right)
	$$
	由于 $x_{-i}^{\prime}=x_{-i}$ ,我们可以简化这个比例为:
	$$
	A\left(x, x^{\prime}\right)=\min \left(1, \frac{p\left(x_i^{\prime} \mid x_{-i}\right)}{p\left(x_i \mid x_{-i}\right)}\right)
	$$
	但是由于 $x_i^{\prime}$ 是直接从分布 $p\left(x_i^{\prime} \mid x_{-i}\right)$ 中抽样出来的,我们知道这个比例总是等于1,因此接受概率总是1.
	
	这就解释了为什么说Gibbs抽样是MH算法的一个特例: Gibbs抽样总是接受新抽样出的样 本,因此它可以被看作是MH算法中建议分布总是等于条件分布,接受概率总是 1 的一个特殊情况.
	
	\section{Example}
	\subsection{AR(1)模型示例}
	考虑一个AR(1)模型:
	\[ y_t = \phi y_{t-1} + \epsilon_t \]
	其中 \( \epsilon_t \sim N(0, \sigma^2) \)。
	
	我们的目标: 给定观测值 \( y_1, y_2, ... y_n \),使用MCMC估计 \( \phi \)。
	
	\begin{tcolorbox}[enhanced,title={步骤}]	
	
	\begin{enumerate}
		\item 基于AR(1)模型定义似然函数 \( p(y|\phi) \)。
		\item 为 \( \phi \) 定义一个先验 \( p(\phi) \)。
		\item 结合似然和先验得到后验 \( p(\phi|y) \)。
		\item 使用Metropolis-Hastings算法(一种MCMC形式)从后验中抽取样本.
	\end{enumerate}
	
	\end{tcolorbox}
	
		%-------------------------------------------
	\newpage
	\begin{appendices}
		\section{R code}
		\begin{lstlisting}[caption={A MCMC example: R code}, label=lst:r_example, xleftmargin=-2em, xrightmargin=0em]
			# Simulate some data
			set.seed(123)
			n <- 100
			phi_true <- 0.5
			sigma_true <- 1
			y <- arima.sim(n=n, list(ar=phi_true, order=c(1,0,0), sd=sigma_true))
			plot(y,type='l')
			# Initialize parameters for Gibbs sampler
			niter <- 10000
			phi_samples <- numeric(niter)
			sigma2_samples <- numeric(niter)
			phi_samples[1] <- 0  # starting value for phi
			sigma2_samples[1] <- 1  # starting value for sigma^2
			
			s2 <- 10  # variance for phi prior
			alpha <- 0.01
			beta <- 0.01
			
			for (i in 2:niter) {
					# Update phi using Metropolis-Hastings
					phi_candidate <- rnorm(1, phi_samples[i-1], 0.1)
					acceptance_ratio <- exp((sum(dnorm(y[-1], mean=phi_candidate*y[-n], sd=sqrt(sigma2_samples[i-1]), log=TRUE)) + dnorm(phi_candidate, 0, sqrt(s2), log=TRUE)) - (sum(dnorm(y[-1], mean=phi_samples[i-1]*y[-n], sd=sqrt(sigma2_samples[i-1]), log=TRUE)) + dnorm(phi_samples[i-1], 0, sqrt(s2), log=TRUE)))
					if (runif(1) < acceptance_ratio) {
						phi_samples[i] <- phi_candidate
					} else {
						phi_samples[i] <- phi_samples[i-1]
					}
					
					# Update sigma^2 using inverse gamma distribution
					residuals <- y[-1] - phi_samples[i] * y[-n]
					alpha_star <- alpha + n/2
					beta_star <- beta + 0.5 * sum(residuals^2)
					sigma2_samples[i] <- 1 / rgamma(1, alpha_star, beta_star)
			}
			
			mean(phi_samples[2000:niter])
			mean(sigma2_samples[2000:niter])
			
			# Plot results
			plot(phi_samples, type="l")
			plot(sigma2_samples, type="l")
		\end{lstlisting}
		
	\end{appendices}
	
	
	
	%参考文献
	%-------------------------------------------
	\newpage
	%\bibliographystyle{plainnat}%
	%\bibliography{refs.bib}
\end{document}