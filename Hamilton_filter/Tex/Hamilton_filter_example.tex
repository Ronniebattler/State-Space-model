\documentclass[UTF8,12pt]{ctexart}
%加载包
\usepackage{ctex}
\usepackage{geometry}%排版
%a4版面,页边距1英寸,showframe 增加边框的参数。
% 设置为A4纸,边距适中模式(永中office)
\geometry{%
	width = 210mm,%
	height = 297mm,
	left = 31.8mm,%
	right = 31.8mm,%
	top = 25.4mm,%
	bottom = 25.4mm%
}

%\hyphenpenalty = 1000% 断字设置,值越大,断字越少。
%\setlength{\parindent}{2em}% 缩进
%\setlength{\parskip}{0.5ex} % 段间距

\usepackage{cite} %引用
\usepackage{amsmath}
\usepackage{amsfonts}
\usepackage{amssymb}%公式
\usepackage{amsthm}%定理环境

%\usepackage{ntheorem}%定理环境,使用这个会使\maketitle版式出问题
\usepackage{bm}%加粗

\usepackage{mathrsfs}
\numberwithin{equation}{section}%对公式以节{section}为基础进行编号.变成(1.1.1)有chapter才有1.1.1,不然只有section是1.1
%\theoremstyle{plain}%定理用latex默认的版式
\newtheorem{thm}{Theorem}[section]
%\theoremstyle{definition}%定义用definition格式
\newtheorem{defn}{Definition}
%\theoremstyle{remark}%用remark格式
\newtheorem{lemma}[thm]{lemma}
\newtheorem{example}{Example}[section]

\usepackage{multirow}%表格列合并宏包,\multirow命令.

\usepackage{tabularx}%表格等宽,\begin{tabularx}命令.

 
%盒子
\usepackage[many]{tcolorbox}    	% for COLORED BOXES (tikz and xcolor included)
\usepackage{setspace}               % for LINE SPACING
\usepackage{multicol}               % for MULTICOLUMNS
%自定义设定		
	\definecolor{main}{HTML}{5989cf}    % setting main color to be used
	\definecolor{sub}{HTML}{cde4ff}     % setting sub color to be used
	
	\newtcolorbox{boxF}{
		colback=blue!5!white,
		enhanced,
		boxrule = 1.5pt, 
		colframe = white, % making the base for dash line
		borderline = {1.5pt}{0pt}{main, dashed} % add "dashed" for dashed line
	}
\tcbuselibrary{skins, breakable}% 支持文本框跨页

\usepackage[english]{babel}% 载入美式英语断字模板

\usepackage{graphicx}
\usepackage{float}
\usepackage{subfigure} %插入多图时用子图显示的宏包

\usepackage{algorithm,algorithmic}%算法

\usepackage{listings}   %代码块
\usepackage{xcolor}
\definecolor{codegreen}{rgb}{0,0.6,0}
\definecolor{codegray}{rgb}{0.5,0.5,0.5}
\definecolor{codepurple}{rgb}{0.58,0,0.82}
\definecolor{backcolour}{rgb}{0.95,0.95,0.92}
%设置代码块
\lstdefinestyle{mystyle}{
	backgroundcolor=\color{backcolour},   
	commentstyle=\color{codegreen},
	keywordstyle=\color{magenta},
	numberstyle=\tiny\color{codegray},
	stringstyle=\color{codepurple},
	basicstyle=\ttfamily\footnotesize,
	breakatwhitespace=false,         
	breaklines=true,                 
	captionpos=b,                    
	keepspaces=true,                 
	numbers=left,                    
	numbersep=5pt,                  
	showspaces=false,                
	showstringspaces=false,
	showtabs=false,                  
	tabsize=2
}

\lstset{style=mystyle,
	language=R,                                       % 设置语言
}

\usepackage{appendix}%附录

\usepackage{hyperref}%可以生成pdf书签,可以跳转
\hypersetup{
	colorlinks=true,
	linkcolor=black,
	citecolor=black,
}%使得目录没有红框 参考文献引用没有颜色

%侧栏笔记
\usepackage{marginnote}
\setlength{\marginparwidth}{2.8cm}%设置宽度
\renewcommand*{\marginfont}{\color{violet}\footnotesize}%fonts
%运用此命令就可加入侧栏笔记\normalmarginpar\marginnote{}

%图注
\usepackage{caption}

%参考文献
\usepackage[round]{natbib}


%画图
\usepackage{tikz}

%标题页
\title{Example of Hamilton Regime Switching Model}
\author{Renhe W.}
\date{ }

%工具
%使用文本框
%\begin{tcolorbox}[enhanced]	\end{tcolorbox}
%代码框
%{\setmainfont{Courier New Bold}                       %设置代码字体                   
%\begin{lstlisting}

%\end{lstlisting}}

%文章开始部分

\begin{document}
	\captionsetup[figure]{labelfont={bf},labelformat={default},labelsep=period,name={图}}%设置图注
	
	\maketitle
	\tableofcontents %目录
	\listoffigures %图片目录
	\listoftables %表格目录
	\newpage
	\kaishu
	
	\section{Hamilton Regime Switching Model}
	
	\subsection{Regime Switching model}
	Hamilton (1989) presents the regime switching model, which is so influential and is one of the main reference paper of so many academic papers. Let $s_t=0,1,2, \ldots, k$ denotes the state variable with $k$ regimes. In case of a two-state regime switching model, $s_t=0$ and $s_t=1$ can be interpreted as the expansion and recession states at time $t$. A $k$-state regime switching linear regression model has the following form.
	$$
	\begin{array}{rlrl}
		y_t=c_{s_t}+\beta_{s_t} x_t+\epsilon_{s_t, t}, & & \epsilon_{s_t, t} & \sim N\left(0, \sigma_{s_t}\right) \\
		& \Downarrow & & \\
		y_t=c_1+\beta_1 x_t+\epsilon_{1, t}, & & \epsilon_{1, t} \sim N\left(0, \sigma_1\right) \\
		y_t=c_2+\beta_2 x_t+\epsilon_{2, t}, & & \epsilon_{2, t} \sim N\left(0, \sigma_2\right) \\
		\cdots & & \\
		y_t=c_k+\beta_k x_t+\epsilon_{k, t}, & & \epsilon_{k, t} \sim N\left(0, \sigma_k\right)
	\end{array}
	$$
	Since $s_t$ can take on $0,1, \ldots, k$, a transition probability matrix is introduced to describe their transitions.
	\subsection{Markov Transition Probability Matrix}
	Each period, the regime or state follows Markov transition probability matrix. Markov means that transition probability depends on not long history of state transitions but only one lag. As examples, two- or three-state transition probability matrix are of the following forms.
	two-state
	$$
	P\left(s_t=j \mid s_{t-1}=i\right)=P_{i j}=\left[\begin{array}{ll}
		p_{00} & p_{01} \\
		p_{10} & p_{11}
	\end{array}\right]
	$$
	three-state
	$$
	P\left(s_t=j \mid s_{t-1}=i\right)=P_{i j}=\left[\begin{array}{lll}
		p_{00} & p_{01} & p_{02} \\
		p_{10} & p_{11} & p_{12} \\
		p_{20} & p_{21} & p_{22}
	\end{array}\right]
	$$
	where $p_{i j}$ is the probability of transitioning from regime $i$ at time $t-1$ to regime $j$ at time $t$.
	
	\subsection{Example}
	
	A two-state regime switching linear regression model and a transition probability matrix are of the following forms.
	$$
	\begin{aligned}
		& y_t=c_{s_t}+\beta_{s_t} x_t+\epsilon_{s_t, t}, \quad \epsilon_{s_t, t} \sim N\left(0, \sigma_{s_t}^2\right) \\
		& P\left(s_t=j \mid s_{t-1}=i\right)=P_{i j}=\left[\begin{array}{ll}
			p_{00} & p_{01} \\
			p_{10} & p_{11}
		\end{array}\right]
	\end{aligned}
	$$
	Here $s_t$ can take on 0 or 1 and $p_{i j}$ is the probability of transitioning from regime $i$ at time $t-1$ to regime $j$ at time $t$.
	
	\subsection{Hamilton Filtering}
	
	Hamilton filter is of the following sequence as we presented it in the previous post.

	1) $t-1$ state (previous state)
	$$
	\xi_{i, t-1}=P\left(s_{t-1}=i \mid \bar{y}_{t-1} ; \theta\right)
	$$

	2) state transition from $i$ to $j$ (state propagation)
	$$p_{i j}$$
	
	3 ) densities under the two regimes at $t$ (data observations and state dependent errors)
	$$
	\eta_{j t}=\frac{1}{\sqrt{2 \pi} \sigma} \exp \left[-\frac{\left(y_t-c_j-\beta_j x_t\right)^2}{2 \sigma_j^2}\right]
	$$

	4) conditional density of the time $t$ observation (combined likelihood with state being collapsed):
	$$
	\begin{aligned}
		f\left(y_t \mid \tilde{y}_{t-1} ; \theta\right) & =\xi_{0, t-1} p_{00} \eta_{0 t}+\xi_{0, t-1} p_{01} \eta_{1 t} \\
		& +\xi_{1, t-1} p_{10} \eta_{0 t}+\xi_{1, t-1} p_{11} \eta_{1 t}
	\end{aligned}
	$$

	5) $t$ posterior state (corrected from previous state)
	$$
	\xi_{j t}=\frac{\sum_{i=0}^1 \xi_{i, t-1} p_{i j} \eta_{j t}}{f\left(y_t \mid \bar{y}_{t-1} ; \theta\right)}
	$$

	6) use posterior state at time $t$ as previous state a time $t+1$ (substitution)
	$$
	\xi_{j t} \rightarrow \xi_{i, t-1}
	$$

	7) iterate 1) $\sim 6$ ) from $t=1$ to $T$
	As a result of executing this iteration, the sample conditional log likelihood of the observed data can be calculated in the following way.
	$$
	\log f\left(\bar{y}_t \mid y_0 ; \theta\right)=\sum_{t=1}^T f\left(y_t \mid \bar{y}_{t-1} ; \theta\right)
	$$
	With this log-likelihood function, we use a numerical optimization to find the best fitting parameter set $(\bar{\theta})$.
	
	\subsection{Numerical Optimization}
	To start iterations of the Hamilton filter, we need to set $\xi_{i, 0}$ and in most cases the unconditional state probabilities are used.
	$$
	\begin{aligned}
		& \xi_{0,0}=\frac{1-p_{11}}{2-p_{00}-p_{11}} \\
		& \xi_{1,0}=1-\xi_{0,0}
	\end{aligned}
	$$
	

	%-------------------------------------------
%	\newpage
%	\begin{appendices}
%		\section{Matlab code}
%
%		
%	\end{appendices}
	
	%参考文献
	%-------------------------------------------
	\newpage
	%\bibliographystyle{plainnat}%
	%\bibliography{refs.bib}
\end{document}